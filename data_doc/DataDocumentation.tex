
% Template for Elsevier CRC journal article
% version 1.2 dated 09 May 2011

% This file (c) 2009-2011 Elsevier Ltd.  Modifications may be freely made,
% provided the edited file is saved under a different name

% This file contains modifications for Procedia Computer Science
% but may easily be adapted to other journals

% Changes since version 1.1
% - added "procedia" option compliant with ecrc.sty version 1.2a
%   (makes the layout approximately the same as the Word CRC template)
% - added example for generating copyright line in abstract

%-----------------------------------------------------------------------------------

%% This template uses the elsarticle.cls document class and the extension package ecrc.sty
%% For full documentation on usage of elsarticle.cls, consult the documentation "elsdoc.pdf"
%% Further resources available at http://www.elsevier.com/latex

%-----------------------------------------------------------------------------------

%%%%%%%%%%%%%%%%%%%%%%%%%%%%%%%%%%%%%%%%%%%%%%%%%%%%%%%%%%%%%%
%%%%%%%%%%%%%%%%%%%%%%%%%%%%%%%%%%%%%%%%%%%%%%%%%%%%%%%%%%%%%%
%%                                                          %%
%% Important note on usage                                  %%
%% -----------------------                                  %%
%% This file should normally be compiled with PDFLaTeX      %%
%% Using standard LaTeX should work but may produce clashes %%
%%                                                          %%
%%%%%%%%%%%%%%%%%%%%%%%%%%%%%%%%%%%%%%%%%%%%%%%%%%%%%%%%%%%%%%
%%%%%%%%%%%%%%%%%%%%%%%%%%%%%%%%%%%%%%%%%%%%%%%%%%%%%%%%%%%%%%

%% The '3p' and 'times' class options of elsarticle are used for Elsevier CRC
%% Add the 'procedia' option to approximate to the Word template
%\documentclass[3p,times,procedia]{elsarticle}
\documentclass[3p,review,times]{elsarticle}		% review increases spacing

%% The `ecrc' package must be called to make the CRC functionality available
%\usepackage{ecrc}

%% The ecrc package defines commands needed for running heads and logos.
%% For running heads, you can set the journal name, the volume, the starting page and the authors

%% set the volume if you know. Otherwise `00'
%\volume{00}

%% set the starting page if not 1
%\firstpage{1}

%% Give the name of the journal
%\journalname{Procedia Computer Science}

%% Give the author list to appear in the running head
%% Example \runauth{C.V. Radhakrishnan et al.}
%\runauth{}

%% The choice of journal logo is determined by the \jid and \jnltitlelogo commands.
%% A user-supplied logo with the name <\jid>logo.pdf will be inserted if present.
%% e.g. if \jid{yspmi} the system will look for a file yspmilogo.pdf
%% Otherwise the content of \jnltitlelogo will be set between horizontal lines as a default logo

%% Give the abbreviation of the Journal.  Contact the journal editorial office if in any doubt
%\jid{procs}

%% Give a short journal name for the dummy logo (if needed)
%\jnltitlelogo{Procedia Computer Science}

%% Provide the copyright line to appear in the abstract
%% Usage:
%   \CopyrightLine[<text-before-year>]{<year>}{<restt-of-the-copyright-text>}
%   \CopyrightLine[Crown copyright]{2011}{Published by Elsevier Ltd.}
%   \CopyrightLine{2011}{Elsevier Ltd. All rights reserved}
%\CopyrightLine{2011}{Published by Elsevier Ltd.}

%% Hereafter the template follows `elsarticle'.
%% For more details see the existing template files elsarticle-template-harv.tex and elsarticle-template-num.tex.

%% Elsevier CRC generally uses a numbered reference style
%% For this, the conventions of elsarticle-template-num.tex should be followed (included below)
%% If using BibTeX, use the style file elsarticle-num.bst

%% End of ecrc-specific commands
%%%%%%%%%%%%%%%%%%%%%%%%%%%%%%%%%%%%%%%%%%%%%%%%%%%%%%%%%%%%%%%%%%%%%%%%%%

%% The amssymb package provides various useful mathematical symbols
\usepackage{amssymb}
%% The amsthm package provides extended theorem environments
\usepackage{amsthm}

%%% Loading additional packages
\usepackage{graphicx}
%%%\usepackage{subfig}
\usepackage[english]{babel}
\usepackage{amsmath}
\usepackage{amsthm}
\usepackage{enumitem}
\usepackage{multicol}
\usepackage{multirow}
\usepackage{mathrsfs}
\usepackage{a4wide}
\usepackage{graphicx}
\usepackage{booktabs}
\usepackage{array}
\usepackage{caption}
\captionsetup[table]{font=sc} % labelfont=bf, 
\captionsetup[figure]{font=sc} % labelfont=bf, 
\usepackage[multiple]{footmisc}    %for multiple footnotes at one reference point
\usepackage{epstopdf} 		% needed to use .eps %might need option --shell-escape (oder -shell-escape?) in old versions 

\usepackage{setspace}		% for changing spacing for certain paragraphs (tex tables)
%\usepackage[onehalfspacing]{setspace} %[singlespacing] or [doublespacing]
\usepackage{subcaption}


%% The lineno packages adds line numbers. Start line numbering with
%% \begin{linenumbers}, end it with \end{linenumbers}. Or switch it on
%% for the whole article with \linenumbers after \end{frontmatter}.
\usepackage{lineno}

%% natbib.sty is loaded by default. However, natbib options can be
%% provided with \biboptions{...} command. Following options are
%% valid:

%%   round  -  round parentheses are used (default)
%%   square -  square brackets are used   [option]
%%   curly  -  curly braces are used      {option}
%%   angle  -  angle brackets are used    <option>
%%   semicolon  -  multiple citations separated by semi-colon
%%   colon  - same as semicolon, an earlier confusion
%%   comma  -  separated by comma
%%   numbers-  selects numerical citations
%%   super  -  numerical citations as superscripts
%%   sort   -  sorts multiple citations according to order in ref. list
%%   sort&compress   -  like sort, but also compresses numerical citations
%%   compress - compresses without sorting
%%


% \biboptions{}

% if you have landscape tables
\usepackage[figuresright]{rotating}

% put your own definitions here:
%   \newcommand{\cZ}{\cal{Z}}
%   \newtheorem{def}{Definition}[section]
%   ...

% add words to TeX's hyphenation exception list
%\hyphenation{author another created financial paper re-commend-ed Post-Script}

% declarations for front matter

\begin{document}

\begin{frontmatter}

%% Title, authors and addresses

%% use the tnoteref command within \title for footnotes;
%% use the tnotetext command for the associated footnote;
%% use the fnref command within \author or \address for footnotes;
%% use the fntext command for the associated footnote;
%% use the corref command within \author for corresponding author footnotes;
%% use the cortext command for the associated footnote;
%% use the ead command for the email address,
%% and the form \ead[url] for the home page:
%%
%% \title{Title\tnoteref{label1}}
%% \tnotetext[label1]{}
%% \author{Name\corref{cor1}\fnref{label2}}
%% \ead{email address}
%% \ead[url]{home page}
%% \fntext[label2]{}
%% \cortext[cor1]{}
%% \address{Address\fnref{label3}}
%% \fntext[label3]{}


%\dochead{}
%% Use \dochead if there is an article header, e.g. \dochead{Short communication}
%% \dochead can also be used to include a conference title, if directed by the editors
%% e.g. \dochead{17th International Conference on Dynamical Processes in Excited States of Solids}

\title{Draft of the Data Report\tnoteref{t1}} %,t2}}
\tnotetext[t1]{MMB forecast platform documentation.}
%\tnotetext[t2]{First draft}

%% use optional labels to link authors explicitly to addresses:
%% \author[label1,label2]{<author name>}
%% \address[label1]{<address>}
%% \address[label2]{<address>}

\author[vw]{V. Wieland \corref{cor1}}

\author[vw]{M. Kuete \corref{cor3}}
\author[vw]{M. Farkas \corref{cor4}}

\cortext[cor1]{Project supervisor}
%\cortext[cor2]{Responsible for drafting and model comparison}
\cortext[cor3]{Responsible for data}
\cortext[cor4]{Responsible for models}


\address[vw]{IMFS, Goethe University Frankfurt, House of Finance, Theodor-W.-Adorno Platz 3, 60629 Frankfurt am Main, Germany}


%\begin{abstract}
%% Text of abstract
%\end{abstract}

%\begin{keyword}
%% keywords here, in the form: keyword \sep keyword

%% PACS codes here, in the form: \PACS code \sep code

%% MSC codes here, in the form: \MSC code \sep code
%% or \MSC[2008] code \sep code (2000 is the default)

%\end{keyword}

\end{frontmatter}


\newpage
%%
%% Start line numbering here if you want
%%
\linenumbers

%% main text

%%%%%%%%%%%%%%%%%%%%%%%%%%%%%%%%%%%%%%%%%%%%%%%%%%%%%

\section{A Real Time Data Set For DSGE-Estimation And Forecasting }
\label{}

\newtheorem{T0}{Theorem}
\newtheorem{D1}{Lemma}


%%%%%%%%%%%%%%%%%%%%%%%%%%%%%%%%%%%%%
%%%%%%%%%%  US VARIABLES      %%%%%%%%%%%%%%%%%%
%%%%%%%%%%%%%%%%%%%%%%%%%%%%%%%%%%%%%
\section{US Variables}
\subsection{Real Output (xgdp\_q\_obs)}
\begin{description}
	\item[(a)] \textbf{Raw Time Series:}
	\begin{enumerate}
		\item ROUTPUT: Real GNP/GDP (Billions of real dollars, seasonally adjusted)
		\begin{itemize}
			\item Source: Philadelphia Fed,
			\item Quarterly observations.
			\item Quarterly vintages from 1965:Q4 to 2014:Q1. Reflect the data available in the middle month of the quarter.
		\end{itemize}			
	\end{enumerate}
	\item[(b)] \textbf{Transformation:}
	\begin{itemize}
		\item xgdp\_q\_obs: Quarterly  Real GNP/GDP Growth
		\item First Difference in log quarterly observations:
		$$xgdp\_q\_obs_{t}=(\ln(ROUTPUT_t)-\ln(ROUTPUT_{t-1}))\times 100$$
		\item Saved in OUTPUT/xgdp\_q\_obs
	\end{itemize}
\end{description}
\subsection{Real Consumption (pcer\_q\_obs)}
\begin{description}
	\item[(a)] \textbf{Raw Time Series:}
	\begin{enumerate}
		\item NCON: Nominal Personal Consumption Expenditures (Billions of real dollars, seasonally adjusted)
		\begin{itemize}
			\item Source: Philadelphia Fed,
			\item Quarterly observations 
			\item Release frequency: quarterly (from 1965:Q4 to 2014:Q1). Quarterly vintages reflect the data available in the middle of the quarter.
		\end{itemize}
		\item PNGP\_J is the deflator (for the definition see Section 5)
	\end{enumerate}
	\item[(b)] \textbf{Transformation:}
	\begin{itemize}
		\item pcer\_q\_obs: Quarterly  Real Personal Consumption Expenditures Growth
		\item  First Difference in log of deflated quarterly observations:
		$$pcer\_q\_obs_{t}=\left(\ln\left(\frac{NCON_t}{PNGP\_J_t}\right)-\ln\left(\frac{NCON_{t-1}}{PNGP\_J_{t-1}}\right)\right)\times 100$$
		\item Saved in PCER/pcer\_q\_obs
	\end{itemize}
\end{description}
\subsection{Real Investment (fpi\_q\_obs)}

\begin{description}
	\item[(a)] \textbf{Raw Time Series:}
	\begin{enumerate}
		\item FPI: Fixed Private Investment\footnote{This title is as given by ALFRED and values are in nominal terms.} (Billions of real dollars, seasonally adjusted annual rate)
		\begin{itemize}
			\item Source: ALFRED/StLouis,
			\item Quarterly observations.
			\item Release frequency: neither monthly nor quarterly.
		\end{itemize}
		\item PNGP\_J is the deflator (for the definition see Section 5)
	\end{enumerate}
	\item[(b)] \textbf{Transformation:}
	\begin{itemize}			
		\item Quarterly vintages reflect the latest ALFRED/StLouis-release before the 15th of the middle month of the quarter.
		\item fpi\_q\_obs: Quarterly Real Fixed Private Investment Growth
		\item First Difference in log of deflated quarterly observations:
		$$fpi\_q\_obs_{t}=\left(\ln\left(\frac{FPI_t}{PNGP\_J_t}\right)-\ln\left(\frac{FPI_{t-1}}{PNGP\_J_{t-1}}\right)\right)\times 100$$
		\item Saved in FPI/fpi\_q\_obs
	\end{itemize}
\end{description}

\subsection{Real Wages (wage\_obs)}
\begin{description}
	\item[(a)] \textbf{Raw Time Series:}
	\begin{enumerate}
		\item WSD: Wage and Salary Disbursements  (Billions of real dollars, seasonally adjusted, at annual rate)
		\begin{itemize}
			\item Source: Philadelphia Fed,
			\item Quarterly observations
			\item Release frequency:Quarterly(from 1965:Q4 to 2014:Q1). Quarterly vintages reflect the data available in the middle of the quarter.
		\end{itemize}
		\item PNGP\_J is the deflator (for the definition see Section 5).
	\end{enumerate}
	
	\item[(b)] \textbf{Transformation:}
	\begin{itemize}
		\item wage\_obs:  Real Wage Growth
		\item First Difference in log of quarterly observations:
		$$wage\_obs=\left(\ln\left(\frac{WSD_t}{PNGP\_J_t}\right)-\ln\left(\frac{WSD_{t-1}}{PNGP\_J_{t-1}}\right)\right)\times 100$$
		\item Saved in WAGE/wage\_obs
	\end{itemize}
	
\end{description}
\subsection{Inflation (pgdp\_q\_obs)}
\begin{description}
	\item[(a)] \textbf{Raw Time Series:}
	\begin{enumerate}
		\item P: Price Index for GNP/GDP. Index level, seasonally adjusted. Base Year: see DATADSGE/BaseIndex
		\begin{itemize}
			\item Source: Philadelphia Fed,
			\item Quarterly observations
			\item Release frequency: Quarterly (from 1965:Q4 to 2014:Q1). Quarterly vintages reflect the data available in the middle of the quarter.
		\end{itemize}
	\end{enumerate}
	\item[(b)] \textbf{Transformation:}
	\begin{itemize}
		\item pgdp\_q\_obs: Quarter-To-Quarter Rate of Inflation
		\item First Difference in log of quarterly observations:
		$$pgdp\_q\_obs=\left(\ln(P_t)-\ln(P_{t-1}\right))\times 100$$
	\end{itemize}
	\item[(c)] Computation of $PNGP\_J$ (column J in MW vintages)
	\begin{itemize}
		\item Computation of $PNGP\_Level$ (column I in MW vintages)
		$$PNGP\_Level_t=\left\{\begin{array}{cc}
		&\ln(P_t)\mbox{ if }t=1\mbox{ which corresponds to}1960Q1\\
		\\
		&PNGP\_Level_{t-1}+pgdp\_q\_obs_t\mbox{ if }t> 1
		\end{array}\right.$$
		\item Finally, $PNGP\_J_t=\exp(PNGP\_Level_t)$
		\item Saved in PGNP/PNGP\_J
	\end{itemize}
\end{description}
\subsection{RFF (rff\_q\_obs)}
\begin{description}
	\item[(a)] \textbf{Raw Time Series:}
	\begin{enumerate}
		\item FEDFUNDS: Effective Federal Funds Rate
		\begin{itemize}
			\item Source: FRED/StLouis
			\item Quarterly observations: from 1954:Q1 to 2014:Q1
			\item Release: Not revised
			\item Quarterly data corresponds to the average of monthly data over months in the quarter. Not Seasonally Adjusted
		\end{itemize} 			
	\end{enumerate}
	\item[(b)] \textbf{Transformation:} Divide annual rate by four: $rff\_q\_obs=FEDFUNDS/4$
	\begin{itemize}
		\item Saved in FEDFUNDS
	\end{itemize}		
\end{description}
\subsection{Hours (hours\_obs)}
\begin{description}
	\item[(a)] \textbf{Raw Time Series:}
	\begin{enumerate}
		\item H: Indexes of Aggregate Weekly Hours, Total. Base Year: see DATADSGE/BaseIndex
		\begin{itemize}
			\item Source: Philadelphia Fed,
			\item Monthly observations: from 1964-Jan onward
			\item Release frequency: Monthly. From 1971:M9 onward					 
		\end{itemize}
		\item CE16OV:Civilian Employment
		\begin{itemize}
			\item Source: ALFRED/StLouis,
			\item Monthly observations. 
			\item Release frequency: neither monthly nor quarterly.
		\end{itemize}
	\end{enumerate}
	\item[(b)] \textbf{Transformation:}
	\begin{itemize}
		\item Hours worked quarterly vintages are obtained by taking the index in the middle month of the quarter. Quarterly observations are averages of the monthly observations in the quarter (HOURS\_Q). For employment, quarterly vintages obtained by considering the latest release before the 15th of the month in the middle of the quarter while quarterly observations are computed by averaging.
		\item Hours Per Capita:
		$$HOURS\_PER\_CAPITA_t=\ln\left(\frac{HOURS\_Q_t}{CE16OV_t}\right)$$
		\item  Finally:
		$$hours\_obs_t=HOURS\_PER\_CAPITA_t-HPTrend(HOURS\_PER\_CAPITA_t, 16000)$$
		\item  Saved in HOURS/hours\_obs
	\end{itemize}			
\end{description}
\subsection{Real Money Balances: M2 (real\_m2\_growth)}

\begin{description}
	\item[(a)] \textbf{Raw Time Series:}
	\begin{enumerate}
		\item M2SL: M2 Money Stock (Billions of real dollars, seasonally adjusted)
		
		\begin{itemize}
			\item Source: ALFRED/StLouis,
			\item Monthly observations: from 1959-Jan to 2014-Feb
			\item Release frequency: Monthly. 
			% \footnote{See the sheet Match\_PHIL\_StLOUIS in M2 for a comparison with Philadelphia Vintages}.
			The series so-obtained is dubbed M2\_Q.
		\end{itemize}
		\item PNGP\_J is the deflator (for the definition see Section 5).
	\end{enumerate}
	\item[(b)] \textbf{Transformation:}
	\begin{itemize}			     
		\item Quarterly vintages obtained by taking the middle month of the quarter and averaging the monthly observations
		\item First Difference in log of quarterly observations:
		$$real\_m2\_growth_{t}=\left(\ln\left(\frac{M2\_Q_t}{PNGP\_J_t}\right)-\ln\left(\frac{M2\_Q_{t-1}}{PNGP\_J_{t-1}}\right)\right)\times 100$$
		\item Saved in 	M2/real\_m2\_growth
	\end{itemize}
\end{description}

\subsection{Credit Spreads: }

\begin{description}
	\item[(a)] \textbf{Raw Time Series:}
	\begin{enumerate}
		\item Moody's Baa (FRB\_H15\_Baa\_monthly): the annualized Moody's Seasoned Baa Corporate Bond Yield
		\begin{itemize}
			\item Source: Board of Governors of the Federal Reserve System
			\item Monthly observations: from 1919-Jan onward
			\item Quarterly observations,FRB\_H15\_Baa\_Q, obtained by averaging the monthly observations
		\end{itemize}
		\item Moody's Aaa (FRB\_H15\_Aaa\_monthly): the annualized Moody's Seasoned Aaa Corporate Bond Yield
		\begin{itemize}
			\item Source: Board of Governors of the Federal Reserve System
			\item Monthly observations: from 1919-Jan up to 2014-May
			\item Quarterly observations,FRB\_H15\_Aaa\_Q, obtained by averaging the monthly observations
		\end{itemize}
		\item TB10YR(FRB\_H15\_Treasury): the 10-Year Treasury Note Yield at Constant Maturity 	
		\begin{itemize}
			\item Source: Board of Governors of the Federal Reserve System
			\item Monthly observations: from 1953-Apr up to 2014-May
			\item Quarterly observations,FRB\_H15\_Treasury\_Q, obtained by averaging the monthly observations
		\end{itemize}
		\item Gilchrist and Zakrajšek(gzspr\_nf) spread index 
		\begin{itemize}
			\item Source: http://www.aeaweb.org/articles.php?doi=10.1257/aer.102.4.1692
			\item Monthly observations: from 1973-Jan up to 2012-Dec
			\item Quarterly observations,GZ\_Q, obtained by averaging the monthly observations
		\end{itemize}		
	\end{enumerate}
	\item[(b)] \textbf{Transformation:}
	\begin{itemize}
		\item Baa\_10YTB: quarterly spread as Moody's Baa over TB10YR, 
		$$Baa\_10YTB=(FRB\_H15\_Baa\_Q - FRB\_H15\_Treasury\_Q)/4 $$
		\item Baa\_Aaa: quarterly spread as Moody's Baa over Aaa
		$$Baa\_Aaa=(FRB\_H15\_Baa\_Q - FRB\_H15\_Aaa\_Q)/4 $$
		\item Baa\_RFF: quarterly spread as Moody's Baa over FEDFUNDS
		$$ Baa\_RFF=(FRB\_H15\_Baa\_Q - FEDFUNDS)/4 $$
		\item Gilchrist and Zakrajšek index:
		$$GZ=GZ\_Q/4$$
	\end{itemize}
\end{description}

\subsection{Mortgage debt by type of holder and property}

\begin{description}
	\item[(a)] \textbf{Raw Time Series:}
	\begin{enumerate}
		\item $MDOTHMFICBTPMFR\_2$ and $MDOTHMFICBTP1T4FR\_2$: Mortgage Debt Outstanding by Type of Holder and Property: Major Financial Institutions: Commercial Banks for Multifamily residences and One-to-four family residences 
		(Millions of Dollars, not seasonally adjusted)
		\item $MDOTHMFIDITPMFR\_2$ and $MDOTHMFIDITP1T4FR\_2$: Mortgage Debt Outstanding by Type of Holder and Property: Major Financial Institutions: Depository institutions for Multifamily residences and One-to-four family residences 
		(Millions of Dollars, not seasonally adjusted)
		
		\begin{itemize}
			\item Source: ALFRED/StLouis,
			\item $MDOTHMFICBTPMFR\_2$, observations: from 1951-Oct to 2013-Oct. Vintages releases: Jan 23th 2014 and March 06th 2014
			\item $MDOTHMFICBTP1T4FR\_2$,observations: from 1951-Oct to 2013-Oct. Vintages releases: Jan 23th 2014 and March 06th 2014
			\item $MDOTHMFIDITPMFR\_2$,  observations: from 1949-Oct to  2014-Jan. Vintages release: June 05th 2014
			\item $MDOTHMFIDITP1T4FR\_2$, ,observations: from 1951-Oct to 2013-Oct. Vintages releases: Jan 23th 2014 and March 06th 2014
			For each of these variables,  the quarterly observations obtained by taking the observation of the first month in the quarter. 
		\end{itemize}
		\item PNGP\_J is the deflator .
	\end{enumerate}
	\item[(b)] \textbf{Transformation:}
	\begin{itemize}
		\item MDOTHMFICB: Mortgage Debt Outstanding by Type of Holder and Property: Commercial banks;
		$$MDOTHMFICB=MDOTHMFICBTPMFR\_2+MDOTHMFICBTP1T4FR\_2$$
		\item  MDOTHMFIDI: Mortgage Debt Outstanding by Type of Holder and Property: Depository Institution;
		$$MDOTHMFIDI=MDOTHMFIDITPMFR\_2+MDOTHMFIDITP1T4FR\_2$$
		
		\item DLNQuaterly 
		$$MDOTHMFICBGROWTH_{t}=\left(\ln\left(\frac{MDOTHMFICB_t}{PNGP\_J_t}\right)-\ln\left(\frac{MDOTHMFICB_{t-1}}{PNGP\_J_{t-1}}\right)\right)\times 100$$ 
		and
		$$MDOTHMFIDIGROWTH_{t}=\left(\ln\left(\frac{MDOTHMFIDI_t}{PNGP\_J_t}\right)-\ln\left(\frac{MDOTHMFIDI_{t-1}}{PNGP\_J_{t-1}}\right)\right)\times 100$$ 
	\end{itemize}
\end{description}
%
%\section{Commercial and industrial loans}
%
%\begin{description}
%\item[(a)] \textbf{Raw Time Series:}
%\begin{enumerate}
%\item $BI1023NCBAM$: Commercial and industrial loans, all commercial banks (Millions of Dollars, seasonally adjusted)
%
%\begin{itemize}
%\item Source: ALFRED/StLouis,
%\item monthly observations: from 1947-Jan to 2014-May
%%\item Quarterly vintages obtained by taking the middle month of the quarter and averaging the monthly observations
%%\footnote{See the sheet Match\_PHIL\_StLOUIS in M2 for a comparison with Philadelphia Vintages}. The series so-obtained is called M2\_Q.
%\end{itemize}
%\item PNGP\_J is the deflator (for the definition see Section 5).
%\end{enumerate}
%\item[(b)] \textbf{Transformation:}
%\begin{itemize}
%\item DLNQuaterly real money balances
%$$real\_m2\_growth_{t}=\left(\ln\left(\frac{M2\_Q_t}{PNGP\_J_t}\right)-\ln\left(\frac{M2\_Q_{t-1}}{PNGP\_J_{t-1}}\right)\right)\times 100$$
%\item Saved in 	M2/real\_m2\_growth
%\end{itemize}
%\end{description}
\subsection{C and I Loans}

\begin{description}
	\item[(a)] \textbf{Raw Time Series:}
	\begin{enumerate}
		\item $EVANQ$: Total Value of Loans for All C and I Loans, All Commercial Banks (Millions of Dollars, not seasonally adjusted)
		
		\begin{itemize}
			\item Source: ALFRED/StLouis,
			\item Quarterly, 1st Full Wk. in 2nd Mo. Of Qtr: from 1997-Apr to 2014-Jan
			\item These data are collected during the middle month of each quarter and are released in the middle of the succeeding month. First vintage release: 2011-03-21. Latest vintage release: 2014-04-01. 
			
			%\footnote{See the sheet Match\_PHIL\_StLOUIS in M2 for a comparison with Philadelphia Vintages}. The series so-obtained is called M2\_Q.
		\end{itemize}
		\item PNGP\_J is the deflator.
	\end{enumerate}
	\item[(b)] \textbf{Transformation:}
	\begin{itemize}			
		\item The vintage of a quarter (EVANQ\_Q) corresponds to the latest release before the 15th of the middle month of the quarter
		\item First Difference in log of quarterly observations:
		$$EVANQ\_QGROWTH_{t}=\left(\ln\left(\frac{EVANQ\_Q_t}{PNGP\_J_t}\right)-\ln\left(\frac{EVANQ\_Q_{t-1}}{PNGP\_J_{t-1}}\right)\right)\times 100$$
		%\item Saved in 	M2/real\_m2\_growth
	\end{itemize}
\end{description}


\section{Christiano Motto Rostagno (2014) dataset}

As in Christiano Motto Rostagno (2014) Risk shock paper the aim is to collect observations on 12 variables. These include the 8 standard variables used in the DNSG14 model \footnote{1. xgdp\_q\_obs,2. pgdp\_a\_obs, 3. rff\_a\_obs, 4. pcer\_q\_obs, 5. fpi\_q\_obs, 6. wage\_obs, 7. hours\_obs, 8.     cp\_q\_obs} and extends it with the following 4 variables:
\begin{enumerate}
	\item Relative price of investment goods (pinv\_q\_obs) ,
	\item Loans to non financial corporations (credit\_q\_obs),
	\item Measure for the slope of the term structure (spreadl\_obs),
	\item Indicator for the entrepreneurial net worth (networth\_q\_obs).
\end{enumerate}
In what follows we discuss the aforementioned 4 series and their construction. Due to data availability the real time data series are substituted for their final, most recent release, as of 2019.01.01.


\subsection{Relative price of investment goods (pinv\_q\_obs)}
\begin{description}
	\item[(a)] \textbf{Raw Time Series:}
	\begin{enumerate}
		\item $A006RD$: Implicit Price Deflators for Gross Domestic Product: Gross private domestic investment - Quarterly
		
		\begin{itemize}
			\item Source: BEA/NIPA-T10109, https://db.nomics.world/BEA/NIPA-T10109
			\item Quarterly, 1st Full Wk. in 2nd Mo. Of Qtr.
			\item Seasonally adjusted
			\item Index 2012=100
			\item 1947-01-01 to 2018-07-01

		
			%\footnote{See the sheet Match\_PHIL\_StLOUIS in M2 for a comparison with Philadelphia Vintages}. The series so-obtained is called M2\_Q.
		\end{itemize}
		\item $A191RD$: Implicit Price Deflators for Gross Domestic Product: Gross private domestic investment price deflator - Quarterly
				\begin{itemize}
					\item Seasonally adjusted
					\item Index 2012=100
					\item 1947-01-01 to 2018-07-01
					\item Note: The final reading of the GDP deflator was used to ensure data consistency.
				\end{itemize}

	\end{enumerate}
	\item[(b)] \textbf{Transformation:}
	\begin{itemize}			
		\item First Difference in log of quarterly observations of the investment deflator devided by the GDP deflator:
		$$pinv\_q\_obs_{t}=\left(\ln\left(\frac{A006RD_t}{A191RD_t}\right)-\ln\left(\frac{A006RD_{t-1}}{A191RD_{t-1}}\right)\right) + 1 $$ 
		%\item Saved in 	M2/real\_m2\_growth
	\end{itemize}
\end{description}

\subsection{Loans to non financial corporations (credit\_q\_obs)}
\begin{description}
	\item[(a)] \textbf{Raw Time Series:}
	\begin{enumerate}
	\item $ Loans to HH: loans\_hh = DBNOMICS: BIS/data/CNFS/Q.US.H.A.M.XDC.A$
    \item $ Loans to NFC: loans\_nfc = DBNOMICS: BIS/data/CNFS/Q.US.H.A.M.XDC.A$
    \item $ US population: pop = DBNOMICS: OECD/MEI/USA.LFWA64TT.STSA.Q$
    \item $ US DGP deflator: A191RD = DBNOMICS: BEA/A191RD$
			
	\end{enumerate}
	\item[(b)] \textbf{Transformation:}

		\begin{align*}
			credit\_q\_obs_{t}&=ln\left( \left(\frac{loans\_hh_t}{pop_t*A191RD_t}\right) + \left(\frac{loans\_nfc_t}{pop_t*A191RD_t}\right) *10^6\right) - \\
		&-	ln\left(\left(\frac{loans\_hh_{t-1}}{pop_{t-1}*A191RD_{t-1}}\right) + \left(\frac{loans\_nfc_{t-1}}{pop_{t-1}*A191RD_{t-1}}\right) *10^6\right) +1;
		%\item Saved in 	M2/real\_m2\_growth
\end{align*} 
\end{description}


\subsection{Measure for the slope of the term structure (spreadl\_obs)}
\begin{description}
	\item[(a)] \textbf{Raw Time Series:}
	\begin{enumerate}
		\item $ Long rate : longrate =DBNOMICS: USA.IRLTLT01.ST.Q Yield 10-year federal government securities$
		\item $ Short rate: shortrate = DBNOMICS: FED/H15/129.FF.O - Qaurterly average of Federal funds – Overnight$
		
	\end{enumerate}
	\item[(b)] \textbf{Transformation:}
	
	\begin{align*}
	spreadl\_obs_t = (longrate_t - shortrate_t) +1
	\end{align*} 
\end{description}

\subsection{Indicator for the entrepreneurial net worth (networth\_q\_obs)}
\begin{description}
	\item[(a)] \textbf{Raw Time Series:}
	\begin{enumerate}
		\item $ nw = Dow Jones Wilshire 5000 index, deflated by the GDP price deflator. Board of Governors of the Federal Reserve System and US Department of Commerce - Bureau of Economic Analysis, Quarterly frequency$
	
	\end{enumerate}
	\item[(b)] \textbf{Transformation:}
	
	\begin{align*}
	networth\_q\_obs_t = ln \left(nw_t\right) - ln \left(nw_{t-1}\right)+1 
	\end{align*} 
\end{description}



%%%%%%%%%%%%%%%%%%%%%%%%%%%%%%%%%%%%%
%%%%%%%%%%  EURO AREA VARIABLES      %%%%%%%%%%%%%%
%%%%%%%%%%%%%%%%%%%%%%%%%%%%%%%%%%%%%
\newpage
\section{Euro Area Variables}
\subsection{Real Output Growth (xgdp\_q\_obs)}
\begin{description}
	\item[(a)] \textbf{Raw Time Series:} 
	\begin{enumerate}
		\item RTD.Q.S0.S.G\_GDPM\_TO\_U.E: Nominal GDP (Seasonally adjusted, not working day adjusted, Gross domestic product at market price - Current prices, Euro)
		\begin{itemize}
			\item Source:  Real Time Database-RTD- (context of Euro Area Business Cycle Network). 
			\item Quarterly observations. 
			\item Release frequency: Quarterly. Quarterly vintages (GDPVint) reflect the latest RTD-release before the 15th of the middle month of the quarter.
			\item DEFLATORVint is the GDP-deflator
		\end{itemize}			
	\end{enumerate}
	\item[(b)] \textbf{Transformation:}
	\begin{itemize}
		\item xgdp\_q\_obs: Quarterly  Real GDP Growth
		\item First Difference in log of quarterly observations:
		$$xgdp\_q\_obs_{t}=(\ln\left(\frac{GDPVint_t}{DEFLATORVint_t}\right)-\ln\left(\frac{GDPVint_{t-1}}{DEFLATORVint_{t-1}}\right))\times 100$$
		%\item Saved in OUTPUT/xgdp\_q\_obs
	\end{itemize}
\end{description}
\subsection{Real Consumption Growth (pcer\_q\_obs)}
\begin{description}
	\item[(a)] \textbf{Raw Time Series:} 
	\begin{enumerate}
		\item RTD.Q.S0.S.G\_FCHI\_TO\_U.E: Private Consumption Nominal (PCN) (Seasonally adjusted, not working day adjusted, Final consumption of households and NPISHs - Current prices, Euro).
		\begin{itemize}
			\item Source: Real Time Database-RTD- (context of Euro Area Business Cycle Network). 
			\item Quarterly observations from 1995:Q1 to 2013:Q4. 
			\item Release frequency: Quarterly. Quarterly vintages (PCNVint) reflect the latest RTD-release before the 15th of the middle month of the quarter.
		\end{itemize}
		\item DEFLATORVint is the GDP-deflator (for the definition see Section 5)
	\end{enumerate}
	\item[(b)] \textbf{Transformation:}
	\begin{itemize}
		\item pcer\_q\_obs: Quarterly  Real Personal Consumption Expenditures Growth
		\item First Difference in log of quarterly observations:
		$$pcer\_q\_obs_{t}=\left(\ln\left(\frac{PCNVint_t}{DEFLATORVint_t}\right)-\ln\left(\frac{PCNVint_{t-1}}{DEFLATORVint_{t-1}}\right)\right)\times 100$$
		%\item Saved in PCER/pcer\_q\_obs
	\end{itemize}
\end{description}
\subsection{Real Investment (fpi\_q\_obs)}

\begin{description}
	\item[(a)] \textbf{Raw Time Series:} 
	\begin{enumerate}
		\item RTD.Q.S0.S.G\_GFCF\_TO\_U.E: Gross Fixed Capital Formation Nominal (Seasonally adjusted, not working day adjusted, Gross fixed capital formation - Current prices, Euro).
		\begin{itemize}
			\item Source: Real Time Database-RTD- (context of Euro Area Business Cycle Network).
			\item Quarterly observations from 1995:Q1 to 2013:Q4 
			\item Release frequency: Quarterly. Quarterly vintages (GFCFVint) reflect the latest RTD-release before the 15th of the middle month of the quarter. 
		\end{itemize}
		\item DEFLATORVint  is the GDP-deflator (for the definition see Section 5).
	\end{enumerate}
	\item[(b)] \textbf{Transformation:}
	\begin{itemize}
		\item fpi\_q\_obs: Quarterly Real Fixed Private Investment Growth
		\item First Difference in log of quarterly observations:
		$$fpi\_q\_obs_{t}=\left(\ln\left(\frac{GFCFVint_t}{DEFLATORVint_t}\right)-\ln\left(\frac{GFCFVint_{t-1}}{DEFLATORVint_{t-1}}\right)\right)\times 100$$
		%\item Saved in FPI/fpi\_q\_obs
	\end{itemize}
\end{description}
%\section{Real Wages (wage\_obs)}
%\begin{description}
%\item[(a)] \textbf{Raw Time Series:} 
%\begin{enumerate}
%\item WSD: Wage and Salary Disbursements  (Billions of real dollars, seasonally adjusted, at annual rate)
%\begin{itemize}
%\item Source: Philadelphia Fed, 
%\item Quarterly observations: from 1965:Q4 to 2014:Q1 
%\item Quarterly vintages reflect the data available in the middle of the quarter. 
%\end{itemize}
%\item PNGP\_J is the deflator (for the definition see Section 5).
%\end{enumerate}
%\item[(b)] \textbf{Transformation:}
%\begin{itemize}
%\item wage\_obs:  Real Wage Growth
%\item DLNQuaterly
%$$wage\_obs=\left(\ln\left(\frac{WSD_t}{PNGP\_J_t}\right)-\ln\left(\frac{WSD_{t-1}}{PNGP\_J_{t-1}}\right)\right)\times 100$$
%\item Saved in WAGE/wage\_obs
%\end{itemize}
%\end{description}
\subsection{Inflation (pgdp\_q\_obs)}
\begin{description}
	\item[(a)] \textbf{Raw Time Series:} 
	\begin{enumerate}
		\item RTD.Q.S0.S.G\_GDPM\_TO\_D.X : Price Index for GDP (Seasonally adjusted, not working day adjusted, Gross domestic product at market price - Deflator, Index). 
		\begin{itemize}
			\item Source: Real Time Database-RTD- (context of Euro Area Business Cycle Network). 
			\item Quarterly observations: from 1995:Q1 to 2013:Q4
			\item Quarterly vintages (DEFLATORVint) reflect the latest RTD-release before the 15th of the middle month of the quarter. 
		\end{itemize} 
	\end{enumerate}
	\item[(b)] \textbf{Transformation:}
	\begin{itemize}
		\item pgdp\_q\_obs: Quarter-To-Quarter Rate of Inflation
		\item DLNQuaterly
		$$pgdp\_q\_obs=\left(\ln(DEFLATORVint_t)-\ln(DEFLATORVint_{t-1}\right))\times 100$$
	\end{itemize}
	%\item[(C)] Computation of $DEFLATORVint\_J$ (column J in MW vintages)
	%\begin{itemize}
	%\item Computation of $DEFLATORVint\_Level$ (column I in MW vintages)
	%$$DEFLATORVint\_Level_t=\left\{\begin{array}{cc}
	%&\ln(P_t)\mbox{ if }t=1\\
	%\\
	%&PNGP\_Level_{t-1}+pgdp\_q\_obs_t\mbox{ if }t> 1			
	%\end{array}\right.$$	where $t=1$ corresponds to the first observation available for the vintage.
	%\item Finally, $DEFLATORVint\_J_t=\exp(DEFLATORVint\_Level_J)$
	%\item Saved in PGNP/PNGP\_J
	%\end{itemize}
\end{description}
\subsection{RFF (rff\_q\_obs)}
\begin{description}
	\item[(a)] \textbf{Raw Time Series:} 
	\begin{enumerate}
		\item RTD.M.S0.N.C\_EONIA.E: Monthly, Neither seasonally nor working day adjusted, Rate - Eonia rate, Euro.
		\begin{itemize}
			\item Source: Real Time Database-RTD- (context of Euro Area Business Cycle Network). 
			\item Monthly observations from 1994Jan to 2014Feb of annual nominal interest rate. 
			\item Quarterly vintages (RFFVint) reflect average of monthly observations-in the quarter- of the latest RTD-release before the 15th of the middle month of the quarter.  
		\end{itemize} 			
	\end{enumerate}
	\item[(b)] \textbf{Transformation:} Divide annual rate by four: $rff\_q\_obs=RFFVint/4$
	%\begin{itemize}
	%\item Saved in FEDFUNDS
	%\end{itemize}
	
	
\end{description}
%\section{Hours (hours\_obs)}
%\begin{description}
%\item[(a)] \textbf{Raw Time Series:} 
%\begin{enumerate}
%\item H: Indexes of Aggregate Weekly Hours, Total. Base Year: see DATADSGE/BaseIndex
%\begin{itemize}
%\item Source: Philadelphia Fed, 
%\item Monthly observations: from 1971:M9 to 2014:M1 
%\item Quarterly vintages obtained by taking the index in the middle month of the quarter and averaging the monthly observations within the quarter (HOURS\_Q). 
%\end{itemize} 
%\item CE16OV:Civilian Employment
%\begin{itemize}
%\item Source: ALFRED/StLouis, 
%\item Monthly observations: from 1961:M9 to 2014:M1 
%\item Quarterly vintages obtained by considering the latest release before the 15th of the month in the middle of the quarter and averaging the monthly observations. 
%\end{itemize}
%\end{enumerate}
%\item[(b)] \textbf{Transformation:}
%\begin{itemize}
%\item Hours Per Capita:
%$$HOURS\_PER\_CAPITA_t=\ln\left(\frac{HOURS\_Q_t}{CE16OV_t}\right)$$ 
%\item  Finally:
%$$hours\_obs_t=HOURS\_PER\_CAPITA_t-HPTrend(HOURS\_PER\_CAPITA_t, 16000)$$
%\item  Saved in HOURS/hours\_obs
%\end{itemize}			 
%\end{description}
\subsection{Real Money Balances: M3 (real\_m3\_growth)}

\begin{description}
	\item[(a)] \textbf{Raw Time Series:} 
	\begin{enumerate}
		\item RTD.M.S0.Y.M\_M3\_V\_NC.E: Monthly, Working day and seasonally adjusted, Monetary aggregate M3, all currencies combined - MFIs, central government and post office giro institutions reporting sector - Euro area counterpart, Non-MFIs excluding central government sector - outstanding amounts at the end of the period (stocks), Euro
		
		\begin{itemize}
			\item Source: Real Time Database-RTD- (context of Euro Area Business Cycle Network). 
			\item Monthly observations: from 1970Jan to 2014Jan 
			\item Quarterly vintages (M3Vint) obtained by taking the latest RTD-release before the 15th of the middle month of the quarter and averaging the monthly observations. 
		\end{itemize}
		\item  DEFLATORVint is the GDP-deflator.
	\end{enumerate}
	\item[(b)] \textbf{Transformation:}
	\begin{itemize}
		\item DLNQuaterly real money balances
		$$real\_m3\_growth_{t}=\left(\ln\left(\frac{M3Vint_t}{DEFLATORVint_t}\right)-\ln\left(\frac{M3Vint_{t-1}}{DEFLATORVint_{t-1}}\right)\right)\times 100$$
		%\item Saved in 	M2/real\_m2\_growth
	\end{itemize}
\end{description}

\newpage
%% References
%%
%% Following citation commands can be used in the body text:
%% Usage of \cite is as follows:
%%   \cite{key}         ==>>  [#]
%%   \cite[chap. 2]{key} ==>> [#, chap. 2]
%%

%% References with BibTeX database:

%% Numbered
%\bibliographystyle{model1-num-names}

%% Numbered without titles
%\bibliographystyle{model1a-num-names}

%% Harvard
%\bibliographystyle{model2-names.bst}\biboptions{authoryear}

%% Vancouver numbered
%\usepackage{numcompress}\bibliographystyle{model3-num-names}

%% Vancouver name/year
%\usepackage{numcompress}\bibliographystyle{model4-names}\biboptions{authoryear}

%% APA style
%\bibliographystyle{model5-names}\biboptions{authoryear}

%% AMA style
%\usepackage{numcompress}\bibliographystyle{model6-num-names}

%% `Elsevier LaTeX' style

\bibliographystyle{plainnat} %elsarticle-num}

\bibliography{References}
\nocite{*}

%% Authors are advised to use a BibTeX database file for their reference list.
%% The provided style file elsarticle-num.bst formats references in the required Procedia style

%% For references without a BibTeX database:

% \begin{thebibliography}{00}

%% \bibitem must have the following form:
%%   \bibitem{key}...
%%

% \bibitem{}

% \end{thebibliography}

\end{document}

%%
%% End of file `ecrc-template.tex'. 